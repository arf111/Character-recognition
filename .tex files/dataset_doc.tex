\documentclass{article}
\usepackage[utf8]{inputenc}
% The documentation should describe about:
% 1. The size (number of images/samples/videos) of the dataset
% 2. Description of the dimensions of the images/videos
% 3. In the case of different classes of images, class-wise distribution of images/videos
% 4. In the case of classification dataset, the distribution of the labels
% 5. Citations of the websites from where you accumulated the data. In case you have acquired data from random
% Internet/real life sources, explicitly mention them.
% 6. Link to the Google drive/dropbox/any cloud service folder (public) where you should store the images (You need
% not submit the images in the above mentioned .zip file, just store them and give us the link).

\begin{document}
\begin{titlepage}
\begin{center}
\line(1,0){300}\\
[0.25in]
\huge{\bfseries Dataset Documentation}\\
[0.25in]
\line(1,0){300}\\
[1.25in]
Raphael Elvis Rozario ID: 150104121\\
K M Arefeen Sultan ID: 150104111\\
Rakib Hossain Ayon ID: 150104120\\
Radwan Rahman ID: 150104123\\
[0.25in]
April 2019
\end{center}
\end{titlepage}
\section{Introduction}
In this dataset\cite{Chars74k:2009}, symbols used in both English and Kannada are available. In the English language, Latin script (excluding accents) and Hindu-Arabic numerals are used. For simplicity it is called the "English" characters set. Dataset consists of:
\section{Dataset Description}
The detailed description about the dataset is stated below.
\begin{itemize}
  \item $64$ classes ($0-9$, $A-Z$, $a-z$)
  \item $7705$ characters obtained from natural images
  \item $3410$ hand drawn characters using a tablet PC
  \item $62992$ synthesised characters from computer fonts
\end{itemize}
This gives a total of over $74$K images (which explains the name of the dataset).
We only uses $64$ classes ($0-9$, $A-Z$, $a-z$) for our project.
Each file has a data structure "list" with these elements:
\begin{enumerate}
\item ALLlabels: class label for each sample
\item ALLnames: sub-directory and name of the image for each sample
\item classlabels: set of labels (classes) in this dataset, coded numerically, e.g. $10=A$, $11=B$, ..., $64=z$
\item classnames: strings of the directories where samples of each class are stored
\item NUMclasses: total number of classes in this dataset
\item TRNind: indexes of the training samples. If $20$ splits are used, this is a matrix of $N$ train samples $X$ $20$
\item TSTind: indexes of the test samples. If $20$ splits are used, this is a matrix of $N$ test samples $X 20$
\item VALind: indexes of the validation samples. If $20$ splits are used, this is a matrix of $N$ validation samples $X 20$
\item TXNind: indexes of the texton samples, i.e., samples used to build the vocabulary with the bag-of-visual-words method. If $20$ splits are used, this is a matrix of $N$ texton samples $X 20$
\end{enumerate}

\bibliography{ref}
\bibliographystyle{acm}
\end{document}
